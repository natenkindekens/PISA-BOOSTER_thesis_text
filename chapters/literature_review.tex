\chapter{Literature review}


\section{Introduction}
The integration of software tools into education is far from new. Over the past fifteen years, digital technologies have become an essential component of teaching and learning at all educational levels.
Some researchers have even described the overwhelming increase of educational software as a kind of \textit{digital pandemic} \cite{dancsa_digital_2023}.
From blended learning environments and AI-powered assistants sush as ChatGPT and GitHub Copilot, to interactive applications in primary schools Automatus, the role of software in education has grown rapidely. \\

However, the notion of a "good" educational software tool remains somewhat vague and often controversial. 
This abundance has motivated many studies aiming to evaluate the impact of educational technology on learning outcomes and to explore how such tools can be improved. \\

In this chapter, we first review existing approaches and software tools that aim to support mathematics education and self-assessment. 
We discuss what has been found to work well, where limitations remain, and how these insights inform the development of our own tool, "PISA-Booster". 
Because education is deeply connected to psychological and cognitive processes, we also introduce key theoretical concepts that are relevant to understanding how learners interact with educational software.
These include principles from metacognition, \myworries{HIER NOG ANDERE PRINCIPLES}

\section{Theoretical Background}

\section{State of the Art}

\