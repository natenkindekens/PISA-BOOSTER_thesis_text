 chapter{Literature review}


\section{Introduction}
The integration of software tools into education is far from new. Over the past fifteen years, digital technologies have become an essential component of teaching and learning at all educational levels.
Some researchers have even described the overwhelming increase of educational software as a kind of \textit{digital pandemic} \cite{dancsa_digital_2023}.
From blended learning environments and AI-powered assistants sush as ChatGPT and GitHub Copilot, to interactive applications in primary schools like Automatus \cite{automatus} or MathGarden \cite{rekentuin}, the role of software in education has grown rapidely. \\

However, the notion of a "good" educational software tool remains somewhat vague and often controversial. 
This abundance has motivated many studies aiming to evaluate the impact of educational technology on learning outcomes and to explore how such tools can be improved. \\

In this chapter, we review existing approaches and software tools that aim to support mathematics education and self-assessment. 
We discuss what has been found to work well, where limitations remain, and how these insights inform the development of our own tool, "PISA-Booster". 
Because education is deeply connected to psychological and cognitive processes, we first introduce key theoretical concepts that are relevant to understanding how learners interact with educational software.
These include principles from metacognition, \myworries{HIER NOG ANDERE PRINCIPLES}

\section{Theoretical Background}
\subsection{Psychological Factors in Arithmetic Learning}
Children's arithmetic learning is influenced by a wide range of cognitive mechanisms.
Broadly these can be divided into two major categories: \textit{domain-specific} cognitive factors and \textit{domain-general} cognitive factors (such as studied by Geary \& Moore, 2016 \cite{geary_chapter_2016} and Vanbist \& De Smedt, 2016 \cite{vanbinst_chapter_2016})
Domain-specific factors refer to skills that are directly tied to numerical processing, such as symbolic numerical magnitude \myworries{LEESTEKEN} the ability to understand and compare quantitites. 
In contrast, domain-general cognitive factors (DGCFs) include processes that are not specific to mathematics but support learning across domains, such as working memory, inhibition, shifting and metacognition. \\

Recent research highlights the importance of both categories in predicting mathematical performance.
For instance, Bellon et al. \cite{bellon_more_2019} demonstrated that arithmetic achievement in primary school children is best explained by a combination of domain-specific and domain-general factors, particulary metacognitive monitoring and executive functions such as updating. 

\subsubsection{Metacognition and Its Role in Arithmetic Learning}
Metacognition was first introduced by Flavell (1979) \cite{flavell_metacognition_1979} and later defined by Brown (1987) \cite{brown_metacognition_1987} to be the awareness and regulation of one's own cognitive processes.
It is often conceptualized as a set of skills that include \textit{planning}, \textit{monitoring} and \textit{evaluating} one's performance (Wang et al. 2021 \cite{wang_skill_2021}).
Planning invloves setting strategies and identifying how to reach a goal; monitoring refers to tracking one's understanding and progress; and evaluating involves judging the effectiveness of one's approach after completing a task.\\

A growing body of research demonstrates that these metacognitive skills are positively related to mathematics performance across age groups and educational contexts (Bellon et al. 2019 \cite{bellon_more_2019}, Wang et al. 2021 \cite{wang_skill_2021}, Fu et al. 2024 \cite{fu_relationship_2025}, Xie et al. 2024 \cite{xie_meta_2024}).
For example, Bellon et al. (2019) \cite{bellon_more_2019} found that metacognitive monitoring uniquely predicted arithmetic accuracy in primary school children.






\subsection{Psychological factors in Digital Learing Environments}
\subsection{Integration: Linking Cognititve and Motivational Perspectives}
\section{State of the Art}

\