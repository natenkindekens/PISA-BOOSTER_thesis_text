\chapter{Problem Statement}

In recent years, numerous digital tools have been developed to support arithmetic learning in primary education. Platforms such as \textit{Automatus}, \textit{Rekentuin}, and \textit{Prodigy Math} have successfully demonstrated how adaptive practice and gamified environments can increase motivation and engagement in mathematics. However, despite their success in promoting practice and fluency, these systems are primarily designed around performance-based repetition and extrinsic rewards. \\ \\
While such approaches improve speed and accuracy, they often overlook an essential aspect of learning — the development of metacognitive skills. Metacognition, or the ability to reflect on and regulate one’s own thinking, has been shown to play a critical role in arithmetic learning and long-term academic achievement \cite{bellon_more_2019, wang_skill_2021, flavell_metacognition_1979}. Yet, current digital learning environments rarely include features that actively strenghtens metacognitive awareness, self-assessment, or reflection on performance. This gap limits the potential of educational technology to support deeper learning and self-regulated development in young learners.
\begin{tcolorbox}[colback=gray!5!white, colframe=gray!70!black, title=\textbf{Problem}]
Most existing digital mathematics learning tools focus primarily on adaptive practice and performance-based gamification. 
They do not explicitly incorporate metacognitive components such as self-assessment, reflection, or awareness of learning progress.
As a result, children may improve in arithmetic fluency without developing the metacognitive insight necessary for independent, self-regulated learning.
\end{tcolorbox}
\begin{tcolorbox}[colback=gray!5!white, colframe=gray!70!black, title=\textbf{Goal}]
\textbf{Design and develop a digital tool to accelerate arithmetic learning through self-assessment and metacognitive support.} 
The goal of this thesis is to create an interactive and adaptive web application that combines engaging arithmetic practice with elements that stimulate metacognitive awareness. 
By integrating reflective feedback, confidence judgments, and personalized progress tracking, the tool aims to help children not only practice arithmetic efficiently but also \textbf{understand how they learn and evaluate their own performance.}
The final design should demonstrate how metacognitive components can be implemented in a child-friendly, motivating, and educationally effective environment.
\end{tcolorbox}
The proposed solution will therefore not compete with existing adaptive tools but complement them by focusing on the learner’s internal cognitive processes rather than external rewards. In doing so, this work aims to contribute to the development of next-generation educational technologies that promote both cognitive performance and metacognitive growth in mathematics learning.
